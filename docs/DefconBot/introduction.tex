\chapter{Introduction}

    Video games have become technologically more advanced over the years - exhibiting realistic graphic capabilities coupled with engaging gameplay. The area of Artificial Intelligence (AI) in video games has also arisen as an important factor in determining the quality of the game by measure of a player's experience. With increasing demand for more realistic computer controlled players, events and opponents capable of exhibiting human-like characteritics, AI has been a significant area of interest for the games industry and academics alike. Several methodologies exist to approach the modelling, scripting and the design of AI for video games, aiming to make game-controlled entities more realistic and intelligent. Behavior Trees attempt to improve upon exisiting AI methodlogies by being simple to implement, scalable to handle the most complex of game tasks and modular to improve reusibility - ultimately improving the efficiency for developing and designing game AI. 

    In this project, we plan to use Behavior Trees to design and develop an AI-controlled player for the commercial real-time strategy game DEFCON. We approach the design of Behavior Trees for the AI using a behavior-oriented methodology which we introduce as Behavior oriented Design (BOD) and also intend for the AI to adapt and learn to play the game of DEFCON by allowing the AI to evolve into a better player automatically using evolutionary machine learning techniques. The project aims to showcase the feasibility of combining such machine learning techniques together with behavior trees as a practical approach to developing an effective and intelligent AI player.

    \newpage
    \section{Motivation}
    
    \subsection{Improvements of Behavior Trees over traditional game AI}
    Behavior Trees have been proposed as a new approach to the designing of game AI, with advantages over other AI approaches being its simple design, scalability and modularity. Behavior Trees have been adopted for use in commercial games for various uses including controlling character animation and coordinating non-playable character(NPC) behaviors. An example of the use of Behavior Trees in commercial game AI includes the AI developed for the commercial First-Person-Shooter(FPS) Halo2~\cite{isla}. Thus, the project wishes to investigate the feasibility of Behavior Trees for the purpose of designing an automated AI player for a real-time strategy(RTS) game such as DEFCON.
    
    \subsection{Complexity of DEFCON as a Real-Time Strategy Game}
    Real-time strategy (RTS) games offer a level of complexity which differ from other games such as platform games, puzzle games or even first-person-shooters. There exists a need to perform micro-managing over the numerous units in the game, as well as adopting strategies to be able to outwit opponents. DEFCON differs from other RTS games as well. In most other RTS games, winning heavily centres around accumulating resource and maximising the size of one's army or number of units. In DEFCON, however, players have an equal set of resources and number of units. The key factor to winning centres around a coming up with a good strategy to coordinate one's units and resources effectively. DEFCON thus exists as a challenging and interesting platform to develop an AI for.
    
    \subsection{Machine Learning Applicabilities in Video Games}
    Previous, the use of machine learning to aid the development of AI in videogames was not commonplace in commercial game development due to factors such as high offline computation and unpredictability of results. However, over time, machine learning techniques have begun to be adopted by commercial game developers after discovering that AI is a necessary ingredient to make games more entertaining and challenging~\cite{ml-and-games}. Several machine learning techniques such as Artifical Neural Networks~\cite{td-gammon} have begun to be used, but several areas such as evolutionary programming, remain less popular due to the vast amount of time and computation required to attain satisfactory results - making them unpractical for games developers and their games - with tight schedules to meet deadlines and requirements for fast, memory efficient computation respectively. 
    
    Thus, we intend to show that evolutionary methods, in particular genetic algorithms, can be used as a practical approach to develop an entertaining, believable and challenging AI and pave the way to its adoption in the future.
    
    \section{Objectives}  
    
    \subsection{Implement Behavior Trees to encode game AI using Behavior Oriented Design as a design methodology}
    
    Exhibit the use of Behavior Trees as a way of developing an AI that is able to handle the complexity of a real-time strategy game. Also, to showcase the Behavior Tree's ease, flexibility and modularity as a means for developing game AI.
    
    \subsection{Research on feasibility of using evolutionary algorithms to develop and improve upon AI}
    
    Demonstrate the ability of game AI to learn to play effectively, and improve upon itself by the use of evolutionary techniques, and thus, the possibility of using such techniques in the development of improved AI in commercial games.
    
    \subsection{Demonstrate the use of reactive planning using Behavior Trees}
    
    Allow the AI developed to accomodate for unpredictability in the game space using reactive planning, allowing it to exhibit intelligent behavior rather than naively following a given plan.
    
    

    

