\begin{abstract}
Writing unit test suites has become an indispensable part of the software engineering process. These test cases, collectively, aim to provide confidence to the client in the final delivery of the shipped product.

Unfortunately, it can be rather tedious to write comprehensive unit tests by hand, especially if the work contributing to this cost is hidden from the customer. Therefore, this project attempts to automate this software testing process, in order to make software development more efficient overall.

Traditionally, the core of software development often revolves around programming in statically typed languages like C/C++, Objective-C or Java/Scala. However, in recent times, it is increasingly common to find their dynamic counterparts, such as Javascript, Python or Ruby, employed in any software stack, and not only just for quick scripting tasks.

This then necessitates the exploration of automated testing tools for dynamic languages, which is the motivation for this project, that is, to automatically generate unit tests using Python, and also address the present inadequate availability of such tools.

In addition, several techniques for accomplishing this have been suggested in papers, although many of which, including \emph{lazy instantiation}, were applied extensively to static languages, like Haskell. Hence, this project intends to research into how a hybrid of these existing ideas could possibly inspire and be adapted to solve the automated unit test generation problem for this domain, in such a way that consistently high code coverage can be achieved across a variety of test programs.
\end{abstract}
